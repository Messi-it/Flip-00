\documentclass[runningheads]{llncs}
\synctex=1

%=================================================================
% 
\newcount\DraftStatus  % 0 suppresses notes to selves in text
\DraftStatus=0   % TODO: set to 0 for final version
%=================================================================

%=================================================================
\usepackage{comment}
%=================================================================
%
\excludecomment{JournalOnly}  
\includecomment{ConferenceOnly}  
\excludecomment{TulipStyle}
%
%=================================================================
\input{preamble}
%
%=================================================================


%=================================================================
%
\begin{document}
%
%=================================================================
% Preamble which will need to be changed for submission
%
\title{TULIP Lab Springer Template: Full paper title}
%
%\titlerunning{TULIP Lab Springer Template}
% If the paper title is too long for the running head, you can set
% an abbreviated paper title here
%
\author{XXX\inst{1}\orcidID{0000-0000-XXXX-0000}
\and
Gang Li\inst{1}\orcidID{0000-0003-1583-641X}\thanks{corresponding author}
}
%
% First names are abbreviated in the running head.
% If there are more than two authors, 'et al.' is used.
\authorrunning{X. YY, G. Li et al.}

\institute{
Deakin University, Geelong, VIC 3216, Australia
\email{gang.li@deakin.edu.au}
\and
School of Computer Science, Xi'an Shiyou University, Shaanxi 710065, China
\email{XXX@tulip.academy}}


%
\maketitle              % typeset the header of the contribution

\begin{abstract}
\blindtext
\keywords{TERM 1 \and  TERM 2 \and TERM 3}
\end{abstract}


%=================================================================

\input{mainbody}

% ----------------------------------------------------------------
\bibliography{tuliplab,yourbib}
\bibliographystyle{splncs04}
%=================================================================

%\listoftodos

\end{document}

